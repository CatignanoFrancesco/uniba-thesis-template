%!TEX program = lualatex

%------------------------------------------------
% PARTE 1 - IMPOSTAZIONI GENERALI DEL DOCUMENTO
% In questa parte ci sono i pacchetti LaTeX
% utilizzati, il carattere e la dimensione
% del testo. Vi è anche la definizione
% linguaggio JSON quando si scrive testo che
% deve essere formattato in questo linguaggio
%------------------------------------------------


\documentclass[10pt, titlepage]{report}


% Pacchetti
\usepackage{graphicx}
\usepackage{float}
\usepackage{listings}
\usepackage{xcolor}
\usepackage[italian]{babel}
\usepackage{subfig}
\usepackage{tabularx}
\usepackage[T1]{fontenc}
\usepackage{afterpage}
\usepackage{multirow}
\usepackage{makecell}


%Impostazioni testo
\usepackage{geometry}
 \geometry{
 	a4paper,
 	left=3cm,
 	right=3cm,
 	top=3cm,
 	bottom=3cm
 }
\usepackage{fontspec}
\setmainfont{Arial}
\linespread{1.5}


%Impostazioni
\newcommand*{\chapterpath}{capitoli}
\graphicspath{ {img/} }

\colorlet{punct}{red!60!black}
\definecolor{background}{HTML}{EEEEEE}
\definecolor{delim}{RGB}{20,105,176}
\colorlet{numb}{magenta!60!black}

% Definizione JSON
\lstdefinelanguage{json}{
	basicstyle=\normalfont\ttfamily,
	numberstyle=\scriptsize,
	stepnumber=1,
	numbersep=4pt,
	showstringspaces=false,
	breaklines=true
}



%--------------------------------------------------
% PARTE 2 - IMPOSTAZIONI SPECIFICHE DEL DOCUMENTO
% In questa parte ci sono il titolo,
% il frontespizio, i capitoli e la bibliografia
%--------------------------------------------------

%Titolo
\title{Titolo della tesi}
\author{Nome e cognome}
\date{01/01/2000}

%Documento
\begin{document}

% Frontespizio
\begin{titlepage}
    \begin{center}
        \includegraphics[width=0.4\textwidth]{logo_uniba}\\
        \vspace{1cm}
        % Dipartimento
        {\large Dipartimento di Inserire il dipartimento}\\
        \vspace{1cm}
        % Corso di laurea
        {\large Corso di laurea in Inserire il corso di laurea}\\
        \hrulefill \\
        \vspace{2cm}
        {\large \textbf{TESI DI LAUREA IN}}\\
        \vspace{0.5cm}
        % Materia
        {\large Materia delle tesi}\\
        \vspace{2cm}
        % Titolo
        {\LARGE\textbf{Inserire qui il titolo della tesi}}\\
        \vspace{2cm}

        \vfill
        
        \begin{tabularx}{\textwidth}{@{}lX}
          & \\
          & {\large Relatore:} \\
          % Relatore
          & {\large \textbf{Prof. Nome Cognome}}
        \end{tabularx}

        \begin{tabularx}{\textwidth}{@{}lX}
            & \\
            & {\large Corelatore:} \\
            % Corelatore
            & {\large \textbf{Prof. Nome Cognome}}
          \end{tabularx}

        \begin{tabularx}{\textwidth}{Xr@{}}
          & \\
          & {\large Laureando:} \\
          % Laureando
          & {\large \textbf{Nome Cognome}}
        \end{tabularx}

        \vspace{1cm}
        \hrulefill \\
        \vspace{1cm}
        % Anno accademico in cui si è iscritti
        {\large Anno Accademico \textbf{2021-2022}}
    \end{center}
\end{titlepage}

% Abstract
\selectlanguage{english}
\renewcommand{\abstractname}{Abstract}  % Per avere il titolo "Abstract" invece di "Sommario"
\include{abstract}
\selectlanguage{italian}

%Indice
\tableofcontents

% Capitoli
\include{capitoli/capitolo1}
\chapter{Immagini, tabelle, codice, liste}
In questo capitolo verranno spiegati tutti i comandi per inserire codice, tabelle di varie dimensioni, elenchi, immagini ecc.

\section{Inserimento di un'immagine}
Durante la produzione del testo può essere utile avere un'immagine o una serie di immagini.
\begin{figure}[H]
	\centering
	\includegraphics[scale=0.4]{image1} % Non è necessario specificare l'estensione. L'immagine viene pescata automaticamente dalla cartella 'img'
	\caption{Immagine singola centrata e scalata}
	\label{capitolo2:image1}
\end{figure}
È possibile fare riferimento alle immagini attraverso la sua label (figura \ref{capitolo2:image1}).\\   % Con questo \\ si va a capo
In altri casi è necessario avere più immagini affiancate.

\end{document}